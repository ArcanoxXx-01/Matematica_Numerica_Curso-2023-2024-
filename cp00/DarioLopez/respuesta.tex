\documentclass[a4paper,12pt]{article}

\begin{document}
\title{CP: 0e100000000000000}
\author{Dario Lopez Falcon}
\date{Enero, 2024}
\maketitle

\section{Ejercicio 1}
\subsection*{Escribe los siguientes números utilizando exactamente la cantidad de caracteres indicados, de
forma que se puedan reconocer como números en el lenguaje de programación de su preferencia2.}

a) 4500000, con 4 caracteres: 45e5

b) -1230000, con 6 caracteres: -123e4

c) 0.000000123, con 6 caracteres: 123e-9

d) 1, con 4 caracteres: -1+2 XD 1e-0

\section*{Ejercicio 2}
En el lenguaje de programación de su preferencia:

\subsection*{ a)Diga cúales de estas igualdades (o desigualdades) son verdaderas}

a) 0.4 * 6 $>$ 2.4 Verdadero si se hace con 0.4 (double) y Falso si se hace con 0.4f (float)


b) 0.8 * 3 == 0.3 * 8  Para 0.8 y 0.3 como double es Falso mientras que si son float es Verdadero.

c) 0.3 * 3 == 0.9 siempre Falso

d ) 3.1 * 2  $<$ 6.2 siempre Falso

e) 1e100 + 1e50 == 1e100 Verdadero :(

\subsection*{ b) Determine cuál es el menor valor natural N para el cual se cumple que:}
1e100 + 10N >1e100.  El menor es 9713344461128646e67 no pude ser mas preciso :(

\subsection*{c) En algunos lenguajes de programación, como C\# o C, es muy fácil trabajar
con distintos tipos de números, por ejemplo, float y double. Resuelva el siguiente ejercicio
para el tipo de datos float y para el tipo de datos double:
}
1e24 + 10N >1e24 El menor es 6710887

\subsection*{d) ¿Qué ocurre si se trata de resolver el inciso b) con el tipo de dato float?}

El menor seria 34028235931503486209399819845154897921f que es MUCHO menor que 9713344461128646e67

\section*{Ejercicio 3}
\subsection*{El objetivo de este ejercicio es tener una primera familiarización con el lenguaje de programaci´on
Python, y una buena forma de hacerlo es implementado algo que sea fácil, por ejemplo, el factorial
de un número :-D. Por eso, lo que hay que hacer es una función en Python que reciba un número
entero y devuelva su factorial. Para ello:}

a) Haga una implementacion recursiva.

b) Haga una implementacion iterativa.

c) Haga una implementacion que use la funcion Gamma que esta implementada
en el modulo scipy.

d) Escriba una variante de la version iterativa en la que, aunque el usuario
pase el argumento como entero, los calculos se realicen con numeros flotantes3

e) ¿A partir de que valor de N los resultados de la variante con enteros es
distinta de la variante con flotantes? 

R/ A partir de 79

\section*{Ejercicio 4}
a) Implemente en Python una funcion que devuelva una aproximacion de la derivada de una
funcion dada, en un punto x usando un parametro h. La funcion implementada debe recibir
tres argumentos f, x, y h, donde f es una funcion f : R → R implementada en Python, x
es un numero real donde se quiere aproximar la derivada de f , y h es el valor del parametro
presente en la aproximacion propuesta.

b) Verifique el buen funcionamiento de su implementacion aproximando la derivada de la funcion
f (x) = x2, en el punto x = 1, con varios valores de h.

c) ¿Para que valor de h se obtienen los mejores resultados en el inciso anterior?

R/Para h=1e-8

\section*{Ejercicio 5}
\subsection*{a) ¿Como se llama esta forma de aproximar una funcion?6}

R/ Serie de Taylor o Polinomio de Taylor

\subsection*{b) ¿Cual es la expresion general para este tipo de aproximacion?}

R/ La misma que el enunciado pero sumando como n-esimo termino o(Xo) que seria o pequeña de Xo

\subsection*{c) Calcule el desarrollo en serie de Taylor de las siguientes funciones, si se desarrollan
alrededor del punto x0 = 0:}

a) f (x) = ex 

b) f (x) = sen(x)

c) f (x) = x5 + 6x3 - 4x2 + 5

\section*{Ejercicio sorpresa}
¿Como se puede saber si la aproximacion con un valor de h es mejor que con otro?

R/ Matematicamente mientras menor sea el h es mas exacto el resultado

\section*{Ejercicio 7}
\subsection*{La derivada de una funcion f : R → R se puede aproximar de varias formas. Algunas de ellas son:}

Poner las ecuaciones (1) y (2)

Tomando como ejemplo las funciones f1(x) = x2 y f2(x) = x3, responda las siguientes preguntas:

a) Si se usa un valor de h = 0,1 en las aproximaciones (1) y (2) para calcular las derivadas de
f1 y f2 en el punto ¯x = 1, ¿exactamente que error se comete en cada caso?

R/Con el primer metodo de derivacion f1 tuvo un error de 2e-15 y f2 tuvo 4e-15

R/Con el segundo metodo de derivacion f1 tuvo un error de 4e-16 y f2 tuvo 16e-16

\section*{Ejercicio 10}
\subsection*{a)}
R/Al parecer es cierto (al menos si se cumplio con la implementacion que hice) :)
\subsection*{b)}
R/Es cierto que funciona pero no es la forma correcta de hacerlo ,a demas si en vez de 
hacer 70 iteraciones hacemos menos ejemplo 10 no cumple
\subsection*{c)}
R/ Es cierto , en las computadoras la suma no es asociatica y lo podemos comprobar muy facil 
haciendo:

a=1e100+1e83+1e83+1e83+1e83+1e83+1e83+1e83+1e83+1e83+1e83

b=1e83+1e83+1e83+1e83+1e83+1e83+1e83+1e83+1e83+1e83+1e100

print(a==b)\#nos va a devolver false

\section*{Ejercicio 12}
\subsection*{Utilizar el metodo de Monte Carlo para aproximar el valor de $\pi$ (100000 boniatos)}


\bibliography{bibliography}
\end{document}